%########################################
% Template in latex for classroom assignments and project report
% Use xelatex compiler 
%########################################



%!TEX TS-program = xelatex
%!TEX encoding = UTF-8 Unicode

\documentclass[10pt,oneside,fleqn]{article}


% Often used text to commands
\newcommand{\clr}{gray!60}
\newcommand{\gray}{\textcolor{gray!60}}
\newcommand{\name}{Le Zhou}
\newcommand{\class}{\gray {Le Zhou}}
\newcommand{\hwnumber}{\gray {}}
\newcommand{\aNum}{A01}
\newcommand{\hw}{\gray {Sensior Design 4013}}



% All packages
\usepackage{amssymb,amsthm,amsmath,enumerate,fancyhdr,graphicx,tabularx}
\usepackage{microtype}
\usepackage{tikz}
\usepackage{graphicx}
\usepackage{pgfplots}
\usepackage{mathpazo}
\usepackage{mdframed}
\usepackage{parskip}
\usepackage{fontspec} 
\usepackage{multicol}
\usepackage{caption}
\usepackage{multicol}
\usepackage{wrapfig}
\usepackage{lipsum}  % generates filler text
\usepackage{url}
\usepackage{hyperref}
\usepackage{float}
\usepackage{xcolor}
\hypersetup{
    colorlinks=true,
    linkcolor=orange,
    filecolor=blue,      
    urlcolor=blue,
    citecolor=cyan,
}
\linespread{1.1} 


% FONTS
\usepackage{xunicode}
\usepackage{xltxtra}
\usepackage[utf8]{inputenc}
\defaultfontfeatures{Mapping=tex-text}
\setromanfont [Ligatures={Common}, Numbers={OldStyle}, Variant=01]{Linux Libertine O}
%\setmonofont[Scale=0.8]{Monaco}
\usepackage{fontspec}
\setmainfont{Linux Libertine O}
\setsansfont{Linux Biolinum O}
%\setmonofont{Inconsolata}

% Frame for Problem Statement definition
%\newenvironment{problem}[1]
%{\begin{mdframed}
%		\textbf{\textsc{Project Objectives:}}
%}
%{\end{mdframed}}



% Environment for Solution definition


% header and footer content
\pagestyle{fancy}
\lhead{\hw {\hwnumber}}
\chead{}
\rhead{\class}
\cfoot{}


% Header line with color gray!60
\renewcommand{\headrulewidth}{0.0pt}% 2pt header rule
\setlength{\headsep}{1.5cm}


\pgfplotsset{compat=1.16}
% Document 


\title{
{\includegraphics[width=0.5\textwidth]{./images/Oklahoma_State_University_Sign.png}}\\
{\large \textbf{General Introduction for Senior Design Ordering }}\\
}
\author{Instructor: Nate Lannan \\
        TA        : Le Zhou}
\date{\today}
	
\begin{document}

\rfoot{\gray \thepage}

\maketitle

\tableofcontents
\newpage
\newpage
Due to the particularity of the senior design, students need to decide on their own the electronic components they need. So we made a systematic introduction on how to buy components.
\section{Items Ordering}
First, we introduce some common the ways you buy electronic components. Based on the \textbf{ OSU system and ECE system}, there are two ways for you to get your items: OK corral system and P-card. 
\subsection{OK corral}
OK corral system is a platform that integrates all kinds of online shopping websites. In the process of making your graduation design, you will often use \url{Mouser.com}, \url{Newegg.com}, and \url{Digikey.com}. On the OK corral system, orders will be sent to a college office. The stuffs who work there will achieve your order. Normally, it will take 3-4 days to receive your order after you send your order to the TA.
\begin{figure}[h]
    \centering
    \includegraphics[width=0.9\textwidth]{images/Picture1.png}
    \caption{Ok corral website.}
    \label{okcorral}
\end{figure} \vspace{-5mm}
\subsection{P-card}
P-card refers to directly purchasing the required components through ECE department. This method is suitable for websites that cannot be found on okcorral, such as \url{amazon.com}.
\section{Documents Sending}
After you have decided on the components you want to buy, you need to fill in the required components in a specific form (Student\_Online\_Order\_Form.xlsx) and then send it to the teaching assistant. The form can be found on your Canvas.
\subsection{Fill In}
You will see some lists in the Student\_Online\_Order\_Form. 
\begin{figure}[H]
    \centering
    \includegraphics[width=0.6\textwidth]{images/figure2.png}
    \caption{Student\_Online\_Order\_Form.}
\end{figure} \vspace{-5mm}
It's easy to understand the meaning of first four lines. The next five lines to fill in are related to the item to be bought. Here I use a component on \url{Mouser.com} as a sample.

\begin{figure}[H]
    \centering
    \includegraphics[width=0.98\textwidth]{images/figure3.png}
    \caption{Information on Mouser.com.}
\end{figure} \vspace{-5mm}
After finding the required information on the form, please fill in the corresponding information in the corresponding position (Figure 4)
\begin{figure}[H]
    \centering
    \includegraphics[width=0.9\textwidth]{images/figure4.png}
    \caption{Information fill in the form.com.}
\end{figure} \vspace{-5mm}
\subsection{Sending}
To send your orders to the TA, please change the file name to \textcolor{red}{team\#\_MM\_DD.xlsx}. And, if any orders are required, please send them to the teaching assistant by \textcolor{teal}{12 noon on Thursdays each week}. As a result, the TA will be able to sort through the orders and deliver them to the school on Thursday afternoon and evening. You will be able to receive the necessary components the following Monday or Tuesday.
\section{Item Receiving}
Since there will be multiple groups of orders arriving at the same time, it takes about half a day for the TA and the storeroom clerk to group the arriving components. Components of each group will be put in different containers and attach labels. When you receive the email, you can make an appointment with the storeroom clerk, and then go to the storage room to take away the components.

\textcolor{teal}{Note: when you get the components of your team, the storeroom clerk/TA will ask everyone to write down the removed items and leave a signature for verification and error correction.}

If you have any questions please contact the instructor/TA. \{nate.lannan@okstate.edu\}/ \{le.zhou@okstate.edu\}
	
\end{document}



